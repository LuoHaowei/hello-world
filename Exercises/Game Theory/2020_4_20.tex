\documentclass{article}
\usepackage{amsmath}
\usepackage{amssymb}
\title{\textbf{Homework\_2}}
\author{Haowei Luo\\2019201050019}
\begin{document}
    \maketitle
    \noindent\Large{\textit{\textbf{{Game Theory for Applied Economists}}}
    
    \noindent\Large{\textbf{{Problem 2.1}}
    
    The parent observes the action $A$ and chooses a bequest $B$ to maximize his/her utility.
    \begin{equation*}
        \mathop{Max} \limits_B V(I_P-B)+k(I_C+B)
    \end{equation*}
    The first order condition can be written as:
    \begin{align*}
        &V^\prime(I_P-B)=k\\
        &B^*=I_P(A)-V^{\prime-1}(k)
    \end{align*}
    
    Then optimize the child's utility, given the parent's reaction is $B^*$:
    \begin{equation*}
        \mathop{Max} \limits_A U(I_C(A)+I_P(A)-V^{\prime-1}(k))
    \end{equation*}
    The first order condition can be written as:
    \begin{equation*}
        U^{\prime}(I_C+B^*)[I_C^{\prime}(A)+I_P^{\prime}(A)]=0
    \end{equation*}
    The utility function $U$ is increasing and strictly concave. Thus $U^{\prime}(I_C+B^*) \neq 0$ and $I_C^{\prime}(A)+I_P^{\prime}(A)=0$, which is also the first order condition maximizing the family's aggregate income.

    \noindent\Large{\textbf{{Problem 2.2}}

    The parent observes the child's action $S$ and chooses $B$ to maximize his/her utility.
    \begin{equation*}
        \mathop{Max} \limits_B V(I_P-B)+k[U_1(I_C-S)+U_2(S+B)]
    \end{equation*}
    The first order condition can be written as:
    \begin{align*}
        V^\prime(I_P-B)&=kU_2^\prime (S+B)\\
        -1<\frac{\mathrm{d} B^*}{\mathrm{d} S}&=\frac{kU_2^{\prime\prime}}{-kU_2^{\prime\prime}-V^{\prime \prime}}<0\\
    \end{align*}
    In words, the more the child's saving is, the less the parent's bequest will be.

    Then optimize the child's utility given $B^*$:
    \begin{equation*}
        \mathop{Max} \limits_S U_1(I_C-S)+U_2(S+B^*)
    \end{equation*}
    The first order condition can be written as:
    \begin{align*}
        &U_1^\prime(I_C-S)=U_2^\prime(S+B^*)(1+\frac{\mathrm{d} B^*}{\mathrm{d} S})\\
        &0<\frac{U_1^\prime(I_C-S)}{U_2^\prime(S+B^*)}=1+\frac{\mathrm{d} B^*}{\mathrm{d} S}<1\\
    \end{align*}
    and $\partial U_1(I_C-S)/\partial S<0$, $\partial U_2(S+B^*)/\partial S>0$. We know that if $S$ were suitably larger, $U_1(I_C-S)+U_2(S+B^)$ would increase and $B$ would decrease. Thus both the child's and the parent's utility would increase.

    \noindent\Large{\textbf{{Problem 2.4}}

    By backwards-induction, given $c_1<R$, partner 2 has to choose a contribution $c_2$ to satisfy $c_1+c_2=R$ or choose a contribution so that $c_1+c_2<R$. For the former choice, the pair of payoff would be $(\delta(V-c_1^2),\delta(V-c_2^2))$. For the latter choice, the pair of payoff would be $(-\delta c_1^2,-\delta c_2^2)$.

    More specifically, when $0 \leqslant c_1<R-\sqrt{V}$, even if partner 2 decides to satisfy $c_1+c_2=R$, the payoff $\delta(V-c_2^2)$ would be negative. So in that condition, partner 2 would choose $c_2=0$, $c_1+c_2<R$. When $R>c_1>R-\sqrt{V}$, $\delta(V-c_2^2)>0>-\delta c_2^2$, so partner 2 would choose $c_2$ so that $c_1+c_2=R$.
    
    For partner 1, to avoid getting the payoff $-\delta c_1^2$, he/she must choose a contribution $R>c_1>R-\sqrt{V}$ or simply $c_1=R$ in period one. If $\delta [V-(R-\sqrt{V})^2]>V-R^2$, partner 1 would choose a $c_1$ as close to $R-\sqrt{V}$ as possible. If $\delta [V-(R-\sqrt{V})^2]<V-R^2$, partner 1 would contribute $R$ in period one and complete the project.

    \noindent\Large{\textbf{{Problem 2.6}}

    For firm 2 and 3 on the second stage:
    \begin{align*}
        &\mathop{Max} \limits_{q_2 \geqslant 0} \pi_2=\mathop{Max} \limits_{q_2 \geqslant 0}[(a-q_1-q_2-q_3)q_2-cq_2]\\
        &\mathop{Max} \limits_{q_3 \geqslant 0} \pi_3=\mathop{Max} \limits_{q_3 \geqslant 0}[(a-q_1-q_2-q_3)q_3-cq_3]\\
        &\left\{
            \begin{array}{rl}
            q_2 &= \frac{a-c-q_1}{3}\\
            q_3 &= \frac{a-c-q_1}{3}\\
            \end{array} \right.
    \end{align*}

    For firm 1 on the first stage:
    \begin{align*}
        \mathop{Max} \pi_1&=\mathop{Max} \{[a-q_1-q_2(q_1)-q_3(q_1)]q_1-cq_1\}\\
        \frac{\partial \pi_1}{\partial q_1}&=0\\
        q_1&=\frac{a-c}{2}\\
        q_2&=q_3=\frac{a-c}{6}
    \end{align*}

    \noindent\Large{\textbf{{Problem 2.13}}

    The firms choose monopoly price $p_i=\frac{a+c}{2}$ to maximize aggregate profits $\pi = (a-p_i)(p_i-c)$. If there was any deviation, they would switch forever to the stage-game Nash equilibrium $p_i=c$ in the aftermath.

    If the monopoly price level can be sustained, each firm will earn profits $\pi_i=\frac{(a-c)^2}{8}$ in every period. And the discounted profits for each firm is $\frac{1}{1-\delta} \frac{(a-c)^2}{8}$ 
    
    If in period t, one firms deviates from the monopoly price level and gets the miximized profits $\frac{(a-c)^2}{4}$, profits for both firms will be 0 in following periods.

    \begin{align*}
        \frac{1}{1-\delta} \frac{(a-c)^2}{8}& \geqslant \frac{(a-c)^2}{4}\\
        \delta & \geqslant \frac{1}{2}
    \end{align*}

    Thus the firms can use trigger strategies to sustain the monopoly price level in a subgame-perfect Nash equilibrium if and only if $\delta \geqslant \frac{1}{2}$.

    \noindent\Large{\textbf{{Problem 2.15}}

    In Cournot equilibrium for firm i:
    \begin{align*}
        \pi_i&=P(Q)q_i-cq_i\\
        &=(a-\sum_{i=1}^n q_i)q_i-cq_i\\
    \end{align*}
    To maximize profits, the first order condition is:
    \begin{align*}
        \frac{\partial \pi_i}{\partial q_i}&=a-\sum_{i=1}^n q_i-q_i-c=0\\
        q_i&=\frac{a-c}{n+1} \ (i=1,2,3 \dots n)\\
        P&=a-\frac{n(a-c)}{n+1}=\frac{a+nc}{n+1}\\
        \pi_i&=\frac{a+nc}{n+1} \cdot \frac{a-c}{n+1}-c \frac{a-c}{n+1}\\
        &=(\frac{a-c}{n+1})^2
    \end{align*}

    If firms choose to produce monopoly quantity:
    \begin{align*}
        \pi&=Q(a-Q)-cQ\\
        Q&=\frac{a-c}{2}\\
        q_i&=\frac{a-c}{2n}\\
        \pi_i&=\frac{(a-c)^2}{4n}
    \end{align*}
    And when firm i decides to deviate from the monopoly quantity:
    \begin{align*}
        \pi_i&=[a-\frac{n-1}{2n}(a-c)-q_i]q_i-cq_i\\
        \frac{\partial \pi_i}{\partial q_i}&=a-\frac{n-1}{2n}(a-c)-q_i-q_i-c=0\\
        q_i&=\frac{(n+1)(a-c)}{4n}\\
        P&=\frac{(n+1)a+(3n-1)c}{4n}\\
    \end{align*}
    so $\pi_i=\frac{(n+1)^2 (a-c)^2}{16n^2}$

    According to the trigger strategy, firm i won't deviate from the monopoly level if 
    \begin{align*}
        &\frac{(n+1)^2 (a-c)^2}{16n^2}+\delta (\frac{a-c}{n+1})^2+\cdots\\ 
        &\leqslant \frac{(a-c)^2}{4n}+\delta \frac{(a-c)^2}{4n}+\cdots\\
        &\frac{(n+1)^2}{16n^2}+\frac{\delta}{1-\delta}\frac{1}{(n+1)^2} \leqslant \frac{1}{1-\delta}\frac{1}{4n}\\
        &\delta \geqslant \frac{4n(n+1)^2-(n+1)^4}{16n^2-(n+1)^4}\\
        &\Longrightarrow \\
        &\delta \geqslant \frac{(n+1)^2}{(n+1)^2+4n}
    \end{align*}





\end{document}

