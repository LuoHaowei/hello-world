\documentclass{article}
\usepackage{amsmath}
\usepackage{amssymb}
\title{Wei, Xie, and Zhang 2017}
\author{Haowei Luo}
\begin{document}
    \maketitle
    Wei, Shang-Jin, Zhuan Xie, and Xiaobo Zhang. 2017. “From ‘Made in China’ to ‘Innovated in China’: Necessity, Prospect, and Challenges.” Journal of Economic Perspectives 31 (1): 49–70. https://doi.org/10.1257/jep.31.1.49.

    Apart from Equatorial Guinea, a country of less than a million people that literally struck oil, no other economy grew as much during the same period. China’s growth performance is clearly spectacular and exceptional.

    But China's economy has reached a crossroads. The annual growth rate has slowed to about 6-7 percent since 2014. Part of the reason for the slowdown could be cyclical, a result of a relatively weak world economy. But a major part of the reason is structural and fundamental.
    
    Growth of the previous 35 years was based on several key factors:
    \begin{itemize}
        \item Market-oriented institutional reforms, including openness to international trade and direct investment
        \item Low wages
        \item A favorable demographic structure
    \end{itemize}
    They are unlikely to provide the same kind of boost going forward.

    China’s investment-to-GDP ratio was already a remarkable 43.3 percent in 2015, it is hard to expect a high growth rate of productivity from continued physical investment.
    \begin{itemize}
        \item Bai and Zhang (2014) estimated that the returns to investment have shown signs of decline since 2008.
        \item Hsieh and Klenow 2009: Some productivity increase could come from reducing resource misallocation, which could be accomplished by further reforms in the factor and product markets, including reforms of state-owned enterprises.
    \end{itemize}
    However, the pace of reform is unlikely to be as aggresive as in the past.
    \begin{itemize}
        \item Interest groups across China now have more means to block reforms than in the past.
        \item Low-hanging fruit in the area of institutional reforms has been picked.
    \end{itemize}
    Productivity growth from this source also faces a limit. Thus, future growth in China has to come mostly from the growth of labor productivity.

    Since the global financial crisis in 2008, the external demand for Chinese products has weakened. Wages in China have meanwhile increased faster than in almost all other major economies. While a strict family planning policy implemented since the early 1980s once produced an unnaturally low birth rate and therefore an unusually favorable dependence ratio for China, the same force has now produced relatively few people entering the labor force today relative to the
    new retirees, hence yielding an unusually unfavorable dependenc ratio.

    China’s firms have to make a tough choice: in, out, up, or down.
    \begin{itemize}
        \item “In” is to move factories to inland areas where the wage is lower than coastal China.
        \item “Out” means engaging in outbound direct investment, combining Chinese know-how with low wages in other countries.
        \item “Up” means innovation and upgrading, so that the firms no longer need to depend on low-paying unskilled labor. 
        \item  “Down” means closing the business.
    \end{itemize}
    Given the pace of convergence within the country and the cost of logistics facing firms inland, "In" is at best a temporary strategy. A decisive factor for China’s economic future is whether its firms can innovate and upgrade and how fast they can do so.

    We study three questions:
    \begin{itemize}
        \item First, how strong is China’s national investment in R\&D? 
        \item Second,what is the growth of innovation by Chinese firms?
        \item Third, because the Chinese economy continues to have a nontrivial share of state-owned enterprises, we investigate possible resource misallocation in the innovation space.
    \end{itemize}
    
    We answer the first one by comparing the Chinese trajectory in recent years with international experiences.

    We answer the second one by making use of data on patents from China State Intellectual Property Office (SIPO), the United States Patent and Trademark Office (USPTO), and World Intellectual Property Office (WIPO). We compare China’s rate of innovation as compared to other BRICS economies and high-income economies.

    As for the third question, we find existence of misallocations in public fiscal resources. Although with more subsidies from the government, SOEs' performance in innovation is lackluster compared to private enterprises. The elasticity of patent filing or patents granted with respect to expenditures on R\&D is significantly higher for private sector firms. Interestingly, we also find that China’s state-owned enterprises often face higher realized tax burdens (the sum of corporate income tax and value-added tax as a share of sales or value added). To improve the efficiency of resource allocation, the direction of policy reforms should perhaps put weight on leveling the playing fields for firms across all ownership types with simultaneous reductions in discretionary subsidies and taxes.
    \section*{Sources of China’s Growth since 1980 and the Moderation of Growth since 2012}
    The rapid growth in the past has been driven by a combination of two sets of factors: a) market-oriented policy reforms
    \begin{itemize}
        \item Let market determined output prices and factor prices replace administrative prices
        \item Introduce and strengthen property rights
        \item Reduce barriers to international trade and investment
    \end{itemize}
    b) economic fundamentals
    \begin{itemize}
        \item A favorable demographic structure
        \item A low initial level of labor cost
    \end{itemize}

    The Chinese growth miracle started with the “rural household responsibility system” in the early 1980s.
    \begin{itemize}
        \item Collective farming and selling all output to a national procurement plan at a price set by the plan (usually substantially below the would-be market price) $\times$
        \item Farmers were granted land user rights and allowed to sell what they produced in excess of the official quota at a market price. \checkmark
    \end{itemize}
    In a few years,
    \begin{itemize}
        \item Agricultural production and rural incomes witnessed a dramatic increase
        \item Hundreds of millions of farmers were released from their land, providing the nonfarm sector with a seemingly unlimited labor supply
    \end{itemize}

    In the 1980s, China’s labor cost was among the lowest in developing countries, lower than in India and the Philippines and indeed lower than 114 out of 138 non-OECD economies. Restricted to living in rural areas by the hukou system, many workers were in township and village-owned enterprises during the 1980s.

    Reallocation of labor
    \begin{itemize}
        \item Low-productivity farm $\times$
        \item High-productivity manufacturing \checkmark
    \end{itemize}

    During the 1990s, most township and village enterprises were privatized, de jure or de facto. By 2011, the township and village enterprise sector had almost disappeared, with employment plummeting from 129 million in 1995 to merely 6 million in 2011. Among state-owned enterprises, which were overwhelmingly in urban areas, employment fell by about half from 113 million in 1995 to 67 million in 2011. The number of state-owned firms declined from 1,084,433 (or 24 percent of the total number of firms) in 1995 to 521,503 (or 3 percent of the total) in 2014, which was part of a deliberate policy of “grasping the large and letting go of the small”.

    Painful in the short run, tens of millions of urban workers had to leave SOEs. Remarkably, the country avoided a big spike in the unemployment rate. The key is that the de facto privatization was accompanied by aggressive reforms to lower entry barriers faced by private sector entrepreneurs. Almost all of the lost jobs in township and village enterprises and state-owned enterprises were offset by new job opportunities in the dynamic private sector. The number of private enterprises increased by nearly five-fold to about 17 million (18,178,921 × .94) in the period 1995–2014. By 2011, 193 million people worked in private enterprises (including self-employed).

    Through this period, the growth in the Chinese economy has become driven overwhelmingly by the growth in the private sector aided by an expansion in the number of entrepreneurs.The manufacturing sector has been growing faster than either the agricultural or service sectors. Wei and Zhang (2011b) have documented two “70 percent rules” using manufacturing firm census data in 1994 and 2005:
    \begin{itemize}
        \item First, approximately 70 percent of the growth in industrial value added came from private sector firms between these two census years.
        \item Second, approximately 70 percent of private sector growth in value added came from growth in the count of new private sector firms (the extensive margin), while the remaining 30 percent came from growth of existing firms (the intensive margin).
    \end{itemize}

    China also carried out a number of other reforms intended to incentivize local governments to pursue growth-friendly policies.
    \begin{itemize}
        \item Under the fiscal arrangement introduced in the early 1980s, local governments and the central government follow a pre-determined revenue formula (though varying across regions as a function of local bargaining power), which stimulates the incentives of local officials to create a more business-friendly environment.
        \item In spite of the political centralization by the Communist Party, the country has implemented a system of fiscal and economic decentralization that grants local governments sufficient decision-making power—and more importantly incentives—to compete with each other. The local economic growth rate is used as a key performance indicator for the career advancement of officials. The delegation of economic policy authority to local governments, which have better knowledge of local information, and competition for investment and tax base among local governments in the Chinese style of federalism have provided a useful check on the temptation of local government officials to expropriate local firms. As a result, firms acquire some de facto security of property rights, even if the formal property rights institutions are problematic.
    \end{itemize}

    China’s government also set up numerous special economic zones and special development zones in the coastal provinces to attract foreign direct investment in the 1980s and 1990s. These zones help the government to meet two challenges. These zones help the government to meet two challenges.
    \begin{itemize}
        \item Public funding for infrastructure was limited, especially in the early days of the reform era. The government was able to concentrate limited public funding to provide adequate roads, power supply, waste treatment and other infrastructure to the firms within the zones, even when it was not able to improve the infrastructure nationally at the same speed.
        \item Policy reforms within these zones were politically easier than doing the same things on a national scale. The success in these zones in terms of economic growth, employment, and tax revenues in turn facilitated similar market-oriented reforms outside the zones. Foreign-invested firms were and continue to be an important channel for transfer of technology and management ideas from advanced economies to China.
        
    \end{itemize}

    China’s integration with the global economy was accelerated after the country joined the World Trade Organization in December 2001. Foreign-invested firms have often accounted for half of the country’s total exports. By 2004, China had come to be known as the World’s Factory, a label describing not only the sheer volume of its cross-border trade, but also the breadth of its sector coverage.

    The proximate drivers of economic growth include improvement in productivity as a crucial component.The increase in productivity stems from innovations within sectors and the reallocation of resources (mainly workers). Sectoral productivity and structural change accounted for 42 and 17 percent of economic growth during 1978–1995.

    For three decades following the start of market-oriented reforms, China appeared to have an inexhaustible amount of “surplus labor”. But signs of labor shortage started to emerge in the first decade of the 2000s. Wages for unskilled workers showed double-digit growth starting in 2003–2004.
    
    Two features of demographic transition have also been a powerful driver of China’s growth in the past three and a half decades.
    \begin{itemize}
        \item A favorable dependency ratio.
        \item Gender ratio imbalance of the premarital cohort
    \end{itemize}
    China’s sharp decline in fertility rate has meant fewer young dependents to support for a given size of the working cohort. The fraction of primeage people in total population rose steadily for three decades, creating an unusually large demographic dividend, which in turn contributed to economic growth. The one-child policy has yielded an unintended consequence in distorting the sex ratio in favor of boys. As the one-child generation enters the marriageable age, young men face a very competitive marriage market. In order to attract potential brides, families with sons choose to work harder, save more, and take on more risks, including exhibiting a higher propensity to be entrepreneurs. The additional growth due to an unbalanced sex ratio is of an immiserizing type: social welfare is likely to have become lower even though the GDP growth accelerated. The fraction of young men who will not get married in the aggregate is determined by the sex ratio, and not by the economy-wide work effort, risk-taking, or GDP growth rate. In this sense, the extra work effort and risk-taking are futile; households collectively would have been willing to give up this part of income growth in exchange for no sex ratio imbalance.

    But starting in 2012, China’s age cohort of 15–60 started to shrink in absolute size. Policy changes to postpone the official retirement age or to encourage more female labor force participation will at best moderate the resulting decline in the workforce. Because the female labor force participation was very high under the central planning regime before the 1980s, higher than most non-Communist countries in the world, such as the United States, Japan, Germany, India, and Indonesia, the participation rate of women in the labor force has in fact come down over time. The recent relaxation of the family planning policy in November 2015 from the limit of one child per couple to two children per couple, while motivated by a desire to improve the demographic pattern for the economy, will make the dependency ratio worse for the next decade-and-a-half rather than better by adding to the number of children without altering the size of the workforce. After all, no couple can produce a 16-year old right away. The sex ratio at birth started to become less unbalanced in 2009, and the contribution to growth from an unbalanced sex ratio will become weaker over time.
    \section*{Evolution of Aggregate Productivity}
    We perform a simple decomposition based on an aggregate production function approach.
    \begin{itemize}
        \item Investment in physical capital has always been important for China’s growth, accounting for 67.9 percent on average throughout this period. The relative share of contribution from physical investment increased to 107 percent after 2009, which resulted from the government stimulus package in response to the global financial crisis.
        \item The contribution from the growth of human capital has been positive, at 12.5 percent during 1999–2008 and 16 percent during 2009–2015. Because of the outsized role of physical investment in the Chinese economy, the contribution of human capital is smaller than what one typically finds from growth
        decomposition for an OECD economy.
        \item The growth of total factor productivity was a major contributor to GDP growth before 2008, often accounting for 20 percent or more to the total growth. (An exception was the period of 1989–1991, a time of domestic political turbulence and international sanctions.)
    \end{itemize}
    Strikingly, the contributions from the growth of total factor productivity have turned persistently negative since 2009. The Chinese government’s response to the global financial crisis that started in 2008 was to encourage physical investment through an aggressive fiscal (and bank lending) program, but there were no ambitious structural reforms pursued during this period that could have raised aggregate efficiency, and yet GDP growth started to moderate after 2012 and this combination is a recipe for negative growth in total factor productivity.

    \begin{itemize}
        \item One way to raise future productivity growth is to pursue more structural reforms. These include removing barriers to labor mobility from rural to urban areas (the hukou system) and leveling uneven access to bank loans by firms of different ownership.
        \item Another way to raise productivity growth is via innovation. Innovation can take the form of creating new products, new ways of using existing products, new designs, new processes for producing existing products that are more efficient and cost-effective, new ways of organizing business, and new ways of branding and marketing the products or services.
    \end{itemize}

    Can China transition from a world assembly line to an innovation powerhouse? Reasons to be skeptical:
    \begin{itemize}
        \item There is no shortage of news stories of intellectual property rights violations by Chinese companies.
        \item There is criticism that the Chinese school system puts too much weight on rote learning and not enough on creative and critical thinking.
    \end{itemize}
    More optimistic examples are available:
    \begin{itemize}
        \item Tencent
        \item Huawei
        \item The world’s first quantum satellite
    \end{itemize}

    From the China Statistical Yearbook on Science and Technology, In 2000, the survey firms collectively spent nearly 20 percent of their technology improvement budget on importing and digesting foreign technology, about 2 percent on buying technologies from other domestic sources, and 78 percent on developing their own in-house technological improvement. Over time, the share of the first item declines, whereas the last two items expand. By 2014, the survey firms collectively spent 11 percent of their technological improvement budget on importing and digesting international technologies, about 5 percent on buying technologies from domestic sources, and the remaining 84 percent on developing their own in-house technological advancement These numbers indicate in an indirect way the improvement in the domestic innovation capacity in China’s manufacturing sector.
    \section*{Research and Development: Investment and People}
    Innovative leaders at both the corporate and national levels tend to invest heavily in R\&D. The United States, Japan, and Germany invested more than 2.7 percent of their GDP in research and development in 2014, which is almost 50 percent more than an average OECD country (about 1.9 percent in 2014), and about three times as much as most developing countries. If China makes the transition to a more innovative economy, a commitment to R\&D spending is needed as well.

    In 1991, when systematic data on this subject started to be collected, China invested 0.7 percent of GDP in R\&D which was much lower than technological leaders like the United States, Japan, and Germany, but not out of line with big developing economies such as India, Brazil, or South Africa. Indeed, because China’s competitiveness at this time was based on exploiting its vast cheap labor and making use of technologies developed elsewhere, domestic research and development and innovation were not an imperative at this time.

    Clearly, higher-income countries tend to have a higher ratio of R\&D spending to GDP. By 2010, China had reached the median value of that ratio. By 2012, its spending had caught up with the OECD average (at 1.88 percent of GDP in that year) even though China’s income level was still less than one-fifth of the OECD average. By 2014, China’s research and development spending had reached 2.05 percent of GDP. From an aggregate R\&D spending viewpoint, China is an overachiever.

    Another indicator of innovation effort is the share of researchers in the population. In 1996, China had 443 researchers per million people. In comparison, the shares for the United States, Japan, and Korea were 3,122, 4,947, and 2,211 per million, respectively. The Chinese ratio in 1996 was comparable to Brazil (420 per million in 2000) and better than India (153 per million in 1996), though much lower than Russia (3,796 per million in 1996). By 2014, the share in China had grown to 1,113 researchers per million population. Because China’s expenditure has grown faster than the number of researchers, R\&D expenditure per researcher has grown over time as well.
    \section*{The Pace of Innovation as Measured by Growth in Patents}
    Not all dimensions of innovation are equally well measured. The output of innovation can take the form of patents, commercial secrets, improvement in business processes or business models, and others. Innovation can also take place in areas outside the commercial space, such as culture. Since innovation in the form of patents is relatively well measured, we will pay special attention to patents by firms. Our conjecture is that innovation across all dimensions is positively correlated.

    The number of Chinese patents has exploded. The number of patent applications filed in China’s State Intellectual Property Office (SIPO) rocketed from 83,045 in 1995 to more than 2.3 million in 2014, at an annual growth rate of 19 percent. In 2011, China overtook the United States as the country with the most patent filings in the world that year (based on data from WIPO).

    What explains the explosion of Chinese patents? Could it be easy approval or low-quality patents? Some straightforward comparisons across countries suggest not.

    There are several ways to answer that question:
    \begin{itemize}
        \item Patent approval rate
        \item Structure of patents
        \item Number of patents granted in foreign countries
    \end{itemize}
    
    One simple metric for judging ease of patent approval is the patent approval rate, ratio of the number of patents granted in year t to the number of patent applications in year t − 1. Based on data from the World Intellectual Property Organization, the patent approval rate in China in recent years is 30–40 percent, which is essentially in the middle of the approval rates across countries. For example, the Chinese approval rate is higher than those in India and Brazil, which are close to 20 percent, but lower than those in the United States and Korea, which are in the range of 50–60 percent. Therefore, the Chinese patent approval ratio does not seem to be unusually high.

    Among the three types of patents (invention, utility model, and design), the fraction of approved invention patents, arguably the most technically intensive category, rose from 8 percent in 1995 to 18 percent in 2014. In 2005, patents granted to foreign applicants accounted for more than 20 percent of China’s total approved patents, but dropped to 7 percent in 2014, suggesting an increasing role of indigenous innovations in the Chinese economy since 2005.

    One way to consider the quality of Chinese patents is to examine patents applied by and granted to Chinese firms in other countries. As noted earlier, the rate of patents approved by China’s patent office grew at an annual rate of 19 percent from 1995 to 2014. During that period, the number of patents granted to Chinese applicants by patent offices in developed countries was rising even faster at 30 percent per year.

    Of particular interest is a comparison of the number of patents granted by the US Patent and Trademark Office (USPTO) to Chinese firms with those to firms from other countries. The number of patents granted by the USPTO to Chinese corporate applicants rose from 62 in 1995 to 7,236 in 2014. The annual growth rate was 21 percent in the first half of the period (1995–2005) but accelerated to 38 percent a year during the latter half of the period (2005–2014). Of the comparison countries—Brazil, Russia, India, South Africa, German, Japan, and Korea—only India had a similar rate of growth in corporate patents in the United States.

    Two natural adjustments are to consider a country’s population size and income level. To this end, we run cross-country regressions with log number of patents granted to applicants from various comparison countries by the US Patent and Trademark Office as the dependent variable. As explanatory variables, we use the log of population, squared log of population, and country times year (country $\times$ year) fixed effects. The estimated coefficients for the interaction term between county and year fixed effects for selected counties can be interpreted as how a given country does relative to the average international experience based on its population size. China shows steady gains in patents even with these adjustments. Of the comparison countries, India also shows gains over time after these adjustments, but Japan, Germany, the Republic of Korea, the Russian Federation, Brazil, and South Africa do not. Overall, Chinese firms collectively do better in their patent count than what the country’s population size and income level would have suggested.

    One can also look at foreign citations of Chinese patents (granted by China’s State Intellectual Property Office). The count of foreign citations of Chinese invention patents grew at the rate of 33 percent a year during 1995–2005, but accelerated to 51 percent a year from 2005 to 2014. The growth of citations of Chinese utility model patents is similar, at 36 percent per year during 1995–2014. After adjusting for population size and income, Chinese firms perform well. This pattern is consistent with international recognition of rising scientific and innovative ideas out of China. 
    
    Overall, not only has the number of Chinese patents exploded, but a variety of comparisons suggest that Chinese patent quality also exhibits a real and robust improvement over time that is quite favorable relative to international experience. There is no reason to be pessimistic about the intrinsic ability for Chinese firms to innovate.

    \section*{Patterns of Innovation Growth}
    Potential drivers of innovation:
    \begin{itemize}
        \item Rise in relevant market size
        \item Industrial competition
        \item Market size
        \item Change in relative prices (such as rising wages)
    \end{itemize}

    We merge the Chinese patent database with the Annual Survey of Industrial Enterprises in China (ASIEC). The ASIEC database covers all the state-owned enterprises and private firms with sales exceeding 5 million yuan from 1998 to 2009, including ownership information. The patent database contains all patents granted by China’s State Intellectual Property Office between 1985 and 2012.

    One pattern that emerges is that state-owned enterprises in general perform worse than private firms in generating patents. During the period 1998–2009, the number of patents granted to private firms in China grew by 35 percent per year, overtaking the number of patents given to state-owned and foreign firms by a comfortable margin. The drop in the share of patents by state-owned enterprises is partly due to the shrinkage of that sector. Clearly, private firms have become the engine of innovation in China.

    Market size has been regarded as a key driver of innovation in the literature. In other words, firms aiming at larger global markets should be more innovative. In past decades, the Chinese economy has become increasingly integrated with the world economy, in particular since China joined the World Trade Organization in 2001. In this data, exporting firms in China are indeed more innovative than nonexporting firms.

    Since 2003, real wages in China have grown by more than 10 percent a year. Some reckon that China has passed the so-called “Lewis turning point,” which means that an era of ultra-low-wage production is over. While patents are rising for both capital- and labor-intensive firms, the fraction of patents granted to labor-intensive firms increased from 55 percent in 1998 to 66 percent in 2009. Rising labor costs may have induced labor-intensive sectors to come up with more innovations to substitute for labor.

    We can connect the discussions on total factor productivity and on innovation. We separate all firms in the ASIEC sample into those with no patents during 1998– 2007, those with a cumulative patent count of 1–4 patents during the same period, and those with a cumulative count of more than 4 patents. We compute the growth of total factor productivity for each individual firm. We find that firm-level productivity tends to grow faster in the group that engages in more innovation. This suggests that to reverse China’s negative levels of total factor productivity, it would be helpful for China to facilitate conditions that expand both the number of firms that engage in innovative activities and the intensity of innovation per innovating firm.
    \section*{Misallocation of Innovation Resources}
    The innovation gap between China and leading advanced economies such as the United States, Japan, and even Korea is still wide. On the list of 2015 Thomson Reuters’ Top 100 Global Innovators, Japanese and US firms lead the way, while no single Chinese firm makes the list. The numbers of US patents received by either Japanese, German, or Korean firms are still more than twice as many as those obtained by Chinese firms in spite of their smaller population size. Part of the gap reflects different stages of development: as we have shown, both investment in R\&D and innovation measured by patent count are strongly positively related to GDP per capita. However, another contributor to the gap could be resource misallocation in the innovation space.

    Most of the surviving state-owned enterprises are relatively big, and are in upstream industries or strategically important sectors. They are typically subject to less competition than private enterprises. Thus, China’s state-owned firms both absorb nontrivial resources, including government subsidies, and still command nontrivial political weights. Part of China’s move to becoming an innovative economy must be to improve the efficiency of resource allocation between state-owned and private firms. 
    
    China’s state-owned firms continue to receive considerable financial support from the government, including access to low-cost bank loans and R\&D subsidies. In the aftermath of both the 1997 Asian financial crisis and the 2008 global financial crisis, the Chinese government launched stimulus packages which often involved credit expansion and which disproportionately went to state-owned enterprises. The more favorable policies and injection of massive stimulus funds have reduced the returns to capital of state-owned enterprises since 2008, caused a decline in their total factor productivity, and provided a lifeline for inefficient zombie firms. The returns to capital of state-owned enterprises are much lower than their private counterparts. Moreover, state-owned enterprises lagged behind private firms in total factor productivity. These patterns suggest a misallocation of government support between state-owned and private enterprises. Government subsidies for research and development can promote firm innovations in China. Government subsides can be defended on the ground that research and development by firms generate positive externalities. Indeed, most advanced countries subsidize research and development as well. The question is not whether subsidies can be justified at all, but rather whether China’s allocation of such subsidies is consistent with economic efficiency.

    Based on simple averages, it would appear that a greater fraction of state-controlled firms are innovative (that is, they have patents) than domestic private sector firms. Indeed, some state-controlled firms receive many patents in a year. But the simple averages are misleading both because state-controlled firms are much larger on average (and larger firms tend to invest more in R\&D), and because they tend to receive more subsidies from various levels of the government. Subsidies from local governments to local government-controlled firms are especially noteworthy.

    We examine firm-level data for evidence of effectiveness of R\&D spending in generating innovations. Based on firms in the ASIEC sample during 2005–2007, for every 10 million yuan of firm-level investment in research and development, domestic private-sector firms and foreign-invested firms generate 6.5 and 7.6 patents, respectively. In comparison, the same investment by state-owned firms yields a more meager 2.2 patents. We may obtain a more informative picture by sorting firms by size and ownership.  In most of the size categories, domestic private sector firms and foreign-invested firms invest more and generate more patents than their state-owned counterparts.

    Several patterns are especially noteworthy:
    \begin{itemize}
        \item First, the returns to research and development spending—as measured by the number of patents per million yuan of research and development spending on the vertical axis—tend to decline with firm size. Because large firms tend to spend more on research and development, this pattern is consistent with the idea that diminishing returns apply to investment in innovation.
        \item Second, across most size deciles, we see that foreign-invested firms and domestic private sector firms tend to have higher returns to investment in research and development.
        \item Third, we do not observe a connection between firm subsidies (relative to sales) and effectiveness at converting research and spending into innovative outcome as measured by patents. Instead, we see that state-controlled firms tend to have much higher subsidies (relative to sales) than either domestic private firms or foreign-invested firms. Interestingly, because small and medium state-owned firms are mostly owned by local governments, they receive more subsidies from the local governments than large state-owned firms.
    \end{itemize} 
    
    In theory, the most productive firms should pursue innovation and less-productive firms should just imitate. Against this theoretical benchmark, they find that less-productive firms in China engage in too much research and development spending, and the more-productive firms may not do enough. Based on their calibrations, if the R\&D misallocation can be reduced to Taiwan's level, the aggregate productivity growth in Chinese manufacturing during 2001–2007 could have grown by about one-third to one-half.

    In sum, there is prima facie evidence that the pattern of subsidies across China’s firms represents resource misallocation. China’s economy-wide innovative outcomes would have been higher if the subsidies were more evenly spread across firm ownership.The sensible policy reforms would be to provide subsidies only in cases where the social returns exceed private returns- positive externalities (such as certain innovative activities)- without regard to firm ownership.

\end{document}