\documentclass[supercite]{HustGraduPaper}
\hypersetup{
colorlinks=true,
linkcolor=black}
\title{浅析居民消费异质性及其对经济增长路径的影响} 
\author{罗浩威} 
\date{\today} 
\school{经济学院} 
\classnum{经济学创新实验班1501} 
\stunum {U201515868} 
\instructor{王健} 
\usepackage{xltxtra}
\usepackage{bm}
\begin{document}
    \maketitle
    \statement
    \clearpage 
    \pagenumbering{Roman}
    
    \begin{cnabstract}{关键词 ;关键词 ;关键词 }
    \ldots
        
    \end{cnabstract}
    
    \begin{enabstract}{Key ; Key ; Key }
    \ldots
        
    \end{enabstract}

    \tableofcontents
    \clearpage
    \pagenumbering{arabic}

    \section{引言}
    \ldots

    \section{异质性消费者理论}
    本文将家庭持有资产分为以现金、支票及商业银行存款为代表的“高流动性、低收益”资产和与之相反的以住房资产为代表的“低流动性、高收益”资产两类,资产流动性的高低之分在于资产变现时交易成本的高低。此外,本文将居民按资产配置结构及消费行为特征分为三类,即W-HtM消费者、P-HtM消费者和N-HtM消费者。HtM消费者(hand-to-mouth consumers)指因缺少高流动性资产而无法跨期动态平滑消费的消费者,此类消费者受到流动性约束,只得将当期收入用于消费开支。W-HtM消费者(wealthy hand-to-mouth consumers)和P-HtM消费者(poor hand-to-mouth consumers)同属于HtM消费者,这两类消费者的共同点在于其资产配置结构中同样缺乏高流动性资产,但W-HtM消费者相比P-HtM消费者拥有丰厚的低流动性资产,而P-HtM消费者缺少甚至没有此类资产。N-HtM消费者(non hand-to-mouth consumers)指持有充足高流动性资产从而满足生命周期-持久收入假说可以平滑消费路径的消费者。
    
    根据上述的分类,本文将建立跨期消费决策模型以说明HtM类消费者消费行为的决定因素,v该模型同样可用于确定居民所处的HtM状态。假设模型中的家庭仅生存于$t=0,1,2$三期,且显然,家庭仅在$t=1,2$时进行消费。设定家庭效用函数为\begin{equation} v_0=u(c_1)+u(c_2) \end{equation}以体现家庭消费时的偏好,$c_t$表示t期的非耐用品消费。假定每一期中的效用函数$u(c_t)$满足$u'>0$,$u''<0$。由于借贷相当于是对高流动性资产的补充,降低家庭在平滑消费时遭遇流动性约束的资产临界值,而效用的主观贴现率主要影响跨期消费的相对价格及消费水平,两者对结论不会产生影响,所以处于简化分析的目的,本文暂不考虑此两种因素。

    假定在期初即$t=0$时,家庭拥有的初始禀赋为$\omega$,可将$\omega$用于$a$、$m_1$两种资产的配置。$a$属于“低流动性,高收益”资产。在$t=2$,家庭要做出消费决策时$a$以资产收益率的形式提供净收益$R$,但在$t=1$,家庭制定消费决策时$a$无法用于平滑当期的消费。$m_1$属于“高流动性,低收益”资产,在$t=1$、$t=2$两期皆可用于家庭的消费,但设定其收益率为$1<R$。假设在进行初期的资产配置后,家庭在$t=1$时获得收入$y_1$,并于此时将$y_1$的一部分用于当期消费,另一部分储蓄为“高流动性,低收益”资产,在$t_2$时家庭获得收入$y_2$并将所持有的全部资产用于消费,即包括$y_2$、$t=1$时储蓄后家庭持有的“高流动性,低收益”资产$m_2$、$t=0$时初始禀赋中“低流动性,高收益”的部分$a$以及家庭所持有资本引致的资本收入。此处$(y_1,y_2)$于$t=0$时已为家庭所知,因此不存在不确定性,则本模型的关键在于分析不同类型消费者在$t=0$时的资产配置决策与$t=1$时的消费-储蓄决策。

    该家庭是HtM消费者或N-HtM消费者取决于$m_2$是否大于零,而该家庭是P-HtM消费者或W-HtM消费者取决于初期禀赋的资产配置中$a$是否大于零。若模型中的家庭为N-HtM消费者,即拥有充足的“高流动性,低收益”资产,在平滑消费时不受到流动性约束,则在$t=1$期消费完后仍可储蓄“高流动性,低收益资产”,可记为$a \geq 0$、$m_2>0$。且由(2.1)式及$u'>0$、$u''<0$可知,如果$c_1 \neq c_2$,假定为$c_1<c_2$,则$t=1$时消费的边际效用$u'$要高于$t=2$期,只需增加$c_1$,即可提高总效用$v_0$,反之亦然,因此对于N-HtM消费者$c_1=c_2$。若模型中的家庭为W-HtM消费者,则应在$t=1$期后拥有数量较多的“低流动性,高收益”资产,但无“高流动性,低收益”资产,可记为$a>0$、$m_2=0$。若模型中的家庭为P-HtM消费者,则在$t=1$期消费后不持有上述的两种资产,可记为$a=0$、$m_2=0$。初期禀赋的资产配置最优化问题可表示为:
    \begin{align*} 
    v_0= & \max_{m_1,a} u(c_1)+u(c_2)\\
    & s.t.\\
    a+m_1= & \omega\\
    c_1+m_2= & y_1+m_1\\
    c_2= & y_2+m_2+Ra\\
    m_1 \geq & 0,a \geq 0 
    \end{align*}
    
    \section{我国消费异质性程度估计}
    \ldots
    
    \section{消费异质性对经济增长路径的影响}
    \ldots
    
    \section{结论与政策建议}
    \ldots
    
    \begin{thankpage}
    \ldots
    \end{thankpage}

    \bibliography{}
    


\end{document}







