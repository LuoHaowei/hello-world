\documentclass[supercite]{HustGraduPaper}
\hypersetup{
colorlinks=true,
linkcolor=black}
\title{浅析居民消费异质性及其对经济增长路径的影响} 
\author{罗浩威} 
\date{\today} 
\school{经济学院} 
\classnum{经济学创新实验班1501} 
\stunum {U201515868} 
\instructor{王健} 
\usepackage{xltxtra}
\usepackage{bm}
\begin{document}
    \maketitle
    \statement
    \clearpage 
    \pagenumbering{Roman}
    
    \begin{cnabstract}{关键词 ;关键词 ;关键词 }
    \ldots
        
    \end{cnabstract}
    
    \begin{enabstract}{Key ; Key ; Key }
    \ldots
        
    \end{enabstract}

    \tableofcontents
    \clearpage
    \pagenumbering{arabic}

    \section{引言}
    \ldots

    \section{异质性消费者理论}
    本文将家庭持有资产分为以现金、支票及商业银行存款为代表的“高流动性、低收益”资产和与之相反的以住房资产为代表的“低流动性、高收益”资产两类,资产流动性的高低之分在于资产变现时交易成本的高低。此外,本文将居民按资产配置结构及消费行为特征分为三类,即W-HtM消费者、P-HtM消费者和N-HtM消费者。HtM消费者(hand-to-mouth consumers)指因缺少高流动性资产而无法跨期动态平滑消费的消费者,此类消费者受到流动性约束,只得将当期收入用于消费开支。W-HtM消费者(wealthy hand-to-mouth consumers)和P-HtM消费者(poor hand-to-mouth consumers)同属于HtM消费者,这两类消费者的共同点在于其资产配置结构中同样缺乏高流动性资产,但W-HtM消费者相比P-HtM消费者拥有丰厚的低流动性资产,而P-HtM消费者缺少甚至没有此类资产。N-HtM消费者(non hand-to-mouth consumers)指持有充足高流动性资产从而满足生命周期-持久收入假说可以平滑消费路径的消费者。
    
    根据上述的分类,本文将建立跨期消费决策模型以说明HtM类消费者消费行为的决定因素,v该模型同样可用于确定居民所处的HtM状态。假设模型中的家庭仅生存于$t=0,1,2$三期,且显然,家庭仅在$t=1,2$时进行消费。设定家庭效用函数为
    \begin{equation} 
    v_0=u(c_1)+u(c_2) 
    \end{equation}
    以体现家庭消费时的偏好,$c_t$表示t期的非耐用品消费。假定每一期中的效用函数$u(c_t)$满足$u'>0$,$u''<0$。由于借贷相当于是对高流动性资产的补充,降低家庭在平滑消费时遭遇流动性约束的资产临界值,而效用的主观贴现率主要影响跨期消费的相对价格及消费水平,两者对结论不会产生影响,所以处于简化分析的目的,本文暂不考虑此两种因素。

    假定在期初即$t=0$时,家庭拥有的初始禀赋为$\omega$,可将$\omega$用于$a$、$m_1$两种资产的配置。$a$属于“低流动性,高收益”资产。在$t=2$,家庭要做出消费决策时$a$以资产收益率的形式提供净收益$R$,但在$t=1$,家庭制定消费决策时$a$无法用于平滑当期的消费。$m_1$属于“高流动性,低收益”资产,在$t=1$、$t=2$两期皆可用于家庭的消费,但设定其收益率为$1<R$。假设在进行初期的资产配置后,家庭在$t=1$时获得收入$y_1$,并于此时将$y_1$的一部分用于当期消费,另一部分储蓄为“高流动性,低收益”资产,在$t_2$时家庭获得收入$y_2$并将所持有的全部资产用于消费,即包括$y_2$、$t=1$时储蓄后家庭持有的“高流动性,低收益”资产$m_2$、$t=0$时初始禀赋中“低流动性,高收益”的部分$a$以及家庭所持有资本引致的资本收入。此处$(y_1,y_2)$于$t=0$时已为家庭所知,因此不存在不确定性,则本模型的关键在于分析不同类型消费者在$t=0$时的资产配置决策与$t=1$时的消费-储蓄决策。

    该家庭是HtM消费者或N-HtM消费者取决于$m_2$是否大于零,而该家庭是P-HtM消费者或W-HtM消费者取决于初期禀赋的资产配置中$a$是否大于零。若模型中的家庭为N-HtM消费者,即拥有充足的“高流动性,低收益”资产,在平滑消费时不受到流动性约束,则在$t=1$期消费完后仍可储蓄“高流动性,低收益资产”,可记为$a \geq 0$、$m_2>0$。且由(2.1)式及$u'>0$、$u''<0$可知,如果$c_1 \neq c_2$,假定为$c_1<c_2$,则$t=1$时消费的边际效用$u'$要高于$t=2$期,只需增加$c_1$,即可提高总效用$v_0$,反之亦然,因此对于N-HtM消费者$c_1=c_2$。若模型中的家庭为W-HtM消费者,则应在$t=1$期后拥有数量较多的“低流动性,高收益”资产,但无“高流动性,低收益”资产,可记为$a>0$、$m_2=0$。若模型中的家庭为P-HtM消费者,则在$t=1$期消费后不持有上述的两种资产,可记为$a=0$、$m_2=0$。初期禀赋的资产配置最优化问题可表示为:
    \begin{align*} 
    v_0= & \max_{m_1,a} u(c_1)+u(c_2)\\
    & s.t.\\
    a+m_1= & \omega\\
    c_1+m_2= & y_1+m_1\\
    c_2= & y_2+m_2+Ra\\
    m_1 \geq & 0,a \geq 0 
    \end{align*}
    以上约束条件分别代表了初期禀赋的资产配置问题和$t=1,2$两期时消费面临的预算约束。求解此最优化问题可得关于$a$的一阶条件为
    \begin{equation}
    u'(c_1)[1+\frac{\partial m_2}{\partial a}] \geq u'(c_2)[R+\frac{\partial m_2}{\partial a}]
    \end{equation}
    且$a=0$时左式严格大于右式。导数${\partial m_2}/{\partial a}$反映了$t=0$时“低流动性,高收益”资产持有量对$t=1$时消费-储蓄决策的影响,而该决策同样可等同于对有约束条件下最优化问题的求解,且在$t=1$时$a$与$m_1$的数额已经确定,则该问题可表示为:
    \begin{align*} 
    v_1(a)= & \max_{c_1,m_2} u(c_1)+u(c_2)\\
    & s.t.\\
    c_1+m_2= & y_1+\omega-a\\
    c_1+m_2= & y_1+m_1\\
    c_2= & y_2+m_2+Ra\\
    m_2 \geq & 0
    \end{align*}
    可求解一阶条件为:
    \begin{equation}
    u'(c_1) \geq u'(c_2)
    \end{equation}
    当$m_2=0$时$u'(c_1)>u'(c_2)$成立。(2.3)式可视为$t=1$期消费-储蓄决策的短期欧拉方程。当$m_2=0$时,$a$对$m_2$的数量并无影响,即${\partial m_2}/{\partial a} = 0$,而$m_2>0$时,该家庭属于N-HtM消费者,则可知对该家庭而言有$u'(c_1)=u'(c_2)$,则由上述两个一阶条件结合可得
    \begin{equation}
    u'(c_1) \geq Ru'(c_2)
    \end{equation}
    即长期欧拉方程。对比长短期欧拉方程可发现,因为家庭仅在$t=0$时有储蓄“低流动性,高收益”资产的机会,所以对家庭而言,在$t=0$期时$t=1,2$期消费的相对价格为$R$,而在$t=1$期时,此相对价格减少为1。由短期欧拉方程可得
    \begin{equation}
    m_2=\max \lbrace \frac{y_1+\omega-y_2-(1+R)a}{2},0 \rbrace
    \end{equation}
    即当$y_1+\omega-y_2-(1+R)a \leq 0$时,$m_2=0$,家庭面临流动性约束且可判断此家庭为HtM类消费者。出于简化分析的目的,可仅考虑$m_2=0$的一充分条件$y_2 \geq y_1+\omega$,此充分条件表示,对于某固定数量的初始禀赋$\omega$,当$y_2$在数量上远胜于$y_1$时,即使将全部的初始禀赋储蓄为“高流动性,低收益”资产用于平滑当期消费,$m_2=0$依然成立,即$m_1=\omega,a=0$仍无法满足$t=1$期的消费需求。

    假定效用函数$u$遵循CES(constant elasticity of substitution)函数形式且跨期替代弹性为$\sigma$,则依据长期欧拉方程可求得
    \begin{equation}
    a=\max \lbrace \frac{R^\sigma (y_1+\omega)-y_2}{R+R^\sigma},0 \rbrace
    \end{equation}
    如前文所述,W-HtM消费者满足$a>0$,结合此式即为
    \begin{equation}
    R>(\frac{y_2}{y_1+\omega})^\frac{1}{\sigma}
    \end{equation}
    相反的,对于P-HtM消费者$R$小于等于右式。简要分析此式可发现,跨期替代弹性越高,家庭越倾向于储蓄更多“低流动性,高收益”资产以获得生命周期内整体消费水平的提高,表现出W-HtM消费者的行为特征的可能性更高。同理资产收益率$R$越高也会使家庭更倾向于忍受$t=1$期时可能遭遇流动性约束从而消费不足的情况。但当$y_2/y_1$过高时,由于家庭已确定$t=2$时将获得大笔收入,因此出于平滑消费的目的,家庭储蓄“低流动性,高收入”资产、延缓消费的动机将大大减弱,更有可能表现出P-HtM消费者的行为特征。
    
    \section{我国消费异质性程度估计}
    对于两种HtM消费者,在跨期预算约束问题中有两个节点会存在边际消费倾向MPC较高的情况,即在$t$期中消费尽本期所持有的“高流动性,低收益”资产并无法变现其他资产以平滑消费路径时,或在$t$期结束时消费者已达到信贷限额。定义在获得此次收入之后下次收入之前的时间间隔为一支付期,在模型中,家庭持有的“高流动性,低收益”资产余额将在每期末被估量,而大多数情况下,统计调查仅报告每支付期内的资产平均余额或仅报告家庭接受采访当日的余额情况,则显然这会导致对家庭HtM消费者状态的误判。例如,考虑对跨期消费决策模型进行连续时间的推广,家庭在每支付期期初获得“高流动性,低收益”资产作为收入,期内消费采取连续且速率均匀稳定的方式。则此种情况下,即使HtM类家庭也将显示为拥有一定数额的甚至超越了信贷限额的“高流动性,低收益”资产。

    若统计调查中“高流动性,低收益”资产余额为支付期内的资产余额平均值,记$y_{it}$为家庭$i$在支付期$t$的收入,记$a_{it}$为该家庭在此支付期内持有的“低流动性,高收益”资产的数额,记$m_{it}$为支付期内“高流动性,低收益”资产余额的均值。首先考虑家庭无借贷无“高流动性,低收益”资产储蓄的情况下在支付期$t$开启时获得$y_{it}$并在支付期内以均匀稳定的速率将收入全部用于消费,此时均值满足$m_{it}=y_{it}/2$。因此当家庭满足
    \begin{equation}
    \begin{aligned}
    a_{it} & \leq 0\\
    0 \leq & m_{it} \leq y_{it}/2
    \end{aligned}
    \end{equation}
    时,可识别其为P-HtM消费者,$a_{it}$为负值的情况发生的频率较低,例如家庭住房资产下跌并低于剩余贷款的价值,此类低流动性资产无法带来高收益且无法变现以平滑消费,因此将$a_{it}$小于零的家庭算作P-HtM。而当家庭满足
    \begin{equation}
    \begin{aligned}
    a_{it} & > 0\\
    0 \leq & m_{it} \leq y_{it}/2
    \end{aligned}
    \end{equation}
    时,可识别其为W-HtM消费者。其次对于借贷限额为$-\underline{m_{it}}<0$的家庭,可假定其在支付期开启时借入现金资产$\underline{m_{it}}$并同时获得收入$y_{it}$,并在支付期内将所持有的全部“高流动性,低收益”资产用于消费,则相当于在期内的每一时刻家庭所拥有的“高流动性,低收益”资产相对于无借贷的情况都要减少$\underline{m_{it}}$,即对于P-HtM消费者
    \begin{equation}
    \begin{aligned}
    a_{it} & \leq 0\\
    m_{it} & \leq 0\\
    m_{it} & \leq \frac{y_{it}}{2} - \underline{m_{it}}
    \end{aligned}
    \end{equation}
    对于W-HtM消费者
    \begin{equation}
    \begin{aligned}
    a_{it} & > 0\\
    m_{it} & \leq 0\\
    m_{it} & \leq \frac{y_{it}}{2} - \underline{m_{it}}
    \end{aligned}
    \end{equation}
    以上对于家庭有无借贷两种情况分类并依据“高流动性,低收益”资产平均余额来判别HtM消费者类型的方法仅提供HtM类家庭数量估计值的下限。例如,存在HtM类家庭在支付期开启时持有“高流动性,低收益”资产,但该家庭选择将全部此类资产和收入全部用于当期消费,则该家庭依然会遭遇流动性约束,但在此方法下并不计入HtM类消费者。同样,对于上述模型中借贷限额为$-\underline{m_{it}}<0$的家庭,当$y_{it}$过高时,依然不能按此类方法计入HtM类消费者。

    \section{消费异质性对经济增长路径的影响}
    基于Diamond模型构建包含“高流动性,低收益”及“低流动性,高收益”两种资产的经济增长模型,借此浅析异质性消费者及家庭资产结构对于经济增长均衡路径的影响。
    
    \section{结论与政策建议}
    \ldots
    
    \begin{thankpage}
    \ldots
    \end{thankpage}

    \bibliography{}
    


\end{document}







