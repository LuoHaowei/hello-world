\documentclass{article}
\usepackage{amsmath}
\usepackage{amssymb}
\title{Zhu (2012)}
\author{Haowei Luo}
\begin{document}
    \maketitle
    Zhu, Xiaodong. “Understanding China’s Growth: Past, Present, and Future.” Journal of Economic Perspectives 26, no. 4 (November 2012): 103–24. https://doi.org/10.1257/jep.26.4.103.
    
    \begin{itemize}
        \item The pace and scale of China’s economic transformation have no historical precedent. In 1978, China was one of the poorest countries in the world. The real per capita GDP in China was only one-fortieth of the U.S. level and one-tenth the Brazilian level. China’s real per capita GDP is now almost one-fifth the U.S. level and at the same level as Brazil.
        \item I will begin by discussing briefly China’s historical growth performance: China went from the world’s leading economic power about 900 years ago to a situation in which it essentially missed the Industrial Revolution and had close-to-zero growth in per capita GDP from 1800 to 1950.
        \item I then present growth accounting results for the period from 1952 to 1978 and the period since 1978.
        \item But the main focus of this paper will be to examine the sources of growth since 1978, the year when China started economic reform. 
        \item I will argue that China’s rapid growth over the last three decades has been driven by productivity growth rather than by capital investment. 
        \item  I will also examine the contributions of sector-level productivity growth, and of resource reallocation across sectors and across firms within a sector, to aggregate productivity growth.
        \item Despite the rapid growth of the last three decades, China’s productivity is still only 13 percent of the U.S. level, which suggests that China still has plenty of room for productivity growth through further economic reforms.
        \item three main data sources: For examining China’s historical performance, I use the data constructed by Madison (2007); for comparing China with other countries, I use the purchasing power parity data from Penn World Table (PWT 7.0); and for detailed growth accounting exercises, I mainly use the data series my coauthor and I constructed for Brandt and Zhu (2010).
        
    \end{itemize}

    \section*{China’s Historical Economic Performance}
    Many historians think that China’s premodern economic performance reached a peak in the Song Dynasty (circa 1200) when China is though to have had the most advanced technologies (Needham and Ronan 1978), the highest iron output (Hartwell 1962), the highest urbanization rate (Chao 1986), and the largest national economy (Madison 2007) in the world. However, China’s per capita GDP stagnated between 1500 and 1800 while Western Europe’s per capita GDP increased steadily. By the end of the fifteenth century, China had already started to fall behind Western Europe, well before the Industrial Revolution occurred in England. Some historians and economists attribute China’s falling behind to the more centralized and inward-looking political systems of the Ming (1368 –1644) and Qing (1644 –1911) dynasties that stifled innovation and commercial activities in China.

    Not all economic historians agree with this explanation. Kenneth Pomeranz (2000) argues in The Great Divergence that in the eighteenth century, living standards and the degree of commercialization in China’s Lower Yangzi region were comparable to those in the richest parts of Europe and that China only started to fall behind Western Europe after the Industrial Revolution in England. Shiue and Keller (2007) provide evidence. Pomeranz attributes the success of the Industrial Revolution to two lucky breaks for England: accesses to coal and colonies.

    In the nineteenth century and the first half of the twentieth century, there is great divergence in economic performance between China and Western Europe. Brandt, Ma, and Rawski (2012) argue that China’s economic failure during this time period was due to an imperial political-institutional system that protected vested interests of elite groups who in turn were resistant to adoptions of new technologies. This imperial system was significantly weakened and eventually collapsed after two Opium Wars and the Sino-Japanese War. A forced opening of China’s borders, and territories and treaty ports being conceded to the West and to Japan brought to China industrial technologies and factories, but continuous civil wars and World War II prevented the industrialization process from gaining much momentum in China until the 1950s.  Industrialization had so little effect and China’s per capita GDP declined between 1800 and 1950.

    \section*{A Growth Accounting Decomposition for Modern China}
    Capital accumulation was the main source of economic growth in the 1952–1978 period while productivity growth has been the main source of growth since then.

    A standard Cobb–Douglas production function:
    \begin{itemize}
        \item $Y$: GDP
        \item $K$: physical capital stock
        \item $L$: labor (number of workers)
        \item $h$: average level of human capital
        \item $A$: total factor productivity (TFP)
        \item $\alpha$: output elasticity of physical capital, which is usually measured by capital’s share of national income
        \item $Pop$: population
    \end{itemize}
    \begin{equation*}
        Y=AK^\alpha(hL)^{1-\alpha}
    \end{equation*}

    Kehoe and Prescott (2002):
    \begin{align*}
        \frac{Y}{Pop}&=\frac{L}{Pop}(\frac{K}{Y})^{\frac{\alpha}{1-\alpha}}hA^{\frac{1}{1-\alpha}}\\
        ln\frac{Y}{Pop}&=ln\frac{L}{Pop}+\frac{\alpha}{1-\alpha}ln\frac{K}{Y}+lnh+\frac{1}{1-\alpha}lnA\\
        \mathrm{d} ln\frac{Y}{Pop}&=\mathrm{d} ln\frac{L}{Pop}+\frac{\alpha}{1-\alpha} \mathrm{d} ln\frac{K}{Y}+ \mathrm{d} lnh+\frac{1}{1-\alpha} \mathrm{d} lnA\\
        &Growth\ rate\ of\ per\ capita\ GDP\\
        &=growth\ rate\ of\ labor\ participation\ rate\\
        &+\alpha/(1-\alpha)growth\ rate\ of\ the\ capital/output\ ratio\\
        &+growth\ rate\ of\ average\ human\ capital\\
        &+1/(1-\alpha)growth\ rate\ of\ total\ factor\ productivity
    \end{align*}
    In this decomposition the contribution of total factor productivity growth is weighted by $1/(1-\alpha)$, taking into account both the direct contribution of total factor productivity and the indirect contribution through its impact on capital accumulation.

    I will set the value of $\alpha$ to 1/2 to match China’s average capital income share as reported in China’s national accounts. In the pre-1978 period, growth was mainly coming from increases in both physical and human capital rather than increases in productive efficiency. Total factor productivity actually deteriorated during this period. Due to the increases in average schooling years, growth of average human capital partially offseting the reduction in total factor productivity. The labor participation rate increased slightly. The most important source of growth was increases in the physical capital/output ratio. Between 1978 and 2007, the physical capital/output ratio remained roughly constant. China’s  capital investment since 1978 has been keeping up with its rapid rate of output growth but not leading it. The average human capital growth rate was lower than that in the pre-1978 period. Demographic factors played a very limited role. Partly due to the one child policy, the labor participation rate grew faster in this period. Total factor productivity grew rapidly, aggregate productivity growth has been the most important source of China’s growth since 1978.
    
    \section*{Government-led Industrialization between 1952 and 1978}
    The Chinese Communist Party government thought the most effective way to speed up the industrialization process was by increasing investment in heavy industries. China’s government limited household consumption and set low prices for agricultural goods so that forced savings and surpluses extracted from the agricultural sector could be used for investment in such industries. This strategy was not sustainable and had grave welfare consequences. The Great Leap Forward failed to raise the GDP growth rate, and had such disruptive effects on agricultural production that a severe famine occurred when China was hit by adverse weather shocks in 1959. The official statistics admit to 15 million deaths and unofficial estimates suggest double that number or more.

    Despite these disastrous results, the Chinese government continued its unbalanced growth strategy. Unfavorable terms of trade were set on farm products, which effectively imposed heavy taxes on farmers. The hukou or household registration system was implemented to keep heavily taxed farmers from leaving rural areas. Farmers were prohibited from engaging in any nonfarm activities. 

    These policies initially helped to ensure the resources to support the capital accumulation in the industrial sector. They also created incentive problems that significantly reduced the productivity of farmers. In the late 1970s, the agricultural sector included more than 70 percent of China’s labor force but was not even able to provide China’s population with 2,300 calories per capita per day. Emergency grain imports were frequently needed to meet food deficits. China’s nonagricultural sector was dominated by the state-owned enterprises in which resource allocation and production activities were carried out according to government plan. Most of them were inefficient, overflowing with redundant workers, and often producing output for which there was no market demand. There were very few firms in the light industries and constant shortages of consumer products.

    \section*{Sectoral Shifts and Productivity Growth Since 1978}
    There was no grand design of systematic reform policies; instead, economic reforms have taken place in a gradual, experimental, and decentralized fashion.

    The “state sector” includes both state-owned enterprises and shareholding companies; and the “nonstate sector” includes domestic private firms, foreign-invested firms, and collective firms in the nonagricultural sector. We include the shareholding companies in the state sector because many of them are former state-owned enterprises that are still controlled by the state. They continue to receive favorable treatment by the state, have easy access to bank credit, and are concentrated in protected industries. In contrast, the collective firms, including those that are controlled by lower-level governments, receive little support from the state.

    Total factor productivity grew rapidly in both the agricultural and the nonstate sectors. Prior to 1998, in particular, the state sector had very low productivity growth rates. After 1998, though, total factor productivity in the state sector grew rapidly. The high rate of productivity growth in agriculture helped to push workers away from jobs in agriculture. The extraordinary increase in the number of workers in nonstate sector was not sufficient to drive down their productivity. It represents the productivity benefits of a sectoral shift away from the agricultural sector to a sector of the economy that could absorb this labor and still generate rapid productivity growth. The state sector’s share of total employment remained remarkably constant from 1978 until 1997. The restructuring of state enterprises circa 1998 led both to a rise in the rate of productivity growth for this sector and also to a decline in its share of China’s labor.
    
    \section*{Productivity Growth in Agriculture and Structural Transformation}
    Since China had experienced recurring food crises before 1978, it is not surprising that its economic reform started in the agricultural sector. There were two important reforms. First, the government increased prices for agricultural goods. Second, the previous “collective farming system” was shifted to the “household-responsibility system”. Under the new system, each farm household was assigned a fixed quota of grains that the household had to sell to the government at official prices. Any extra grain the household produced could be sold at market prices. Most of the productivity growth in the agricultural sector during this period can be attributed to strong positive incentive effects. The increase in food availability alleviated China’s subsistence food constraint and started a structural transformation that reallocated a large amount of labor from agriculture to industry, especially to rural industrial enterprises.
    
    By about 1984 these static efficiency gains, from workers using the same technology with a much more rewarding set of incentives, were largely exhausted. Both agricultural productivity and structural transformation stagnated in the second half of the 1980s. Starting around 1990, markets for agricultural inputs and outputs were gradually liberalized and government interventions were significantly reduced which provided farmers with strong incentives to adopt new technologies. Most of agriculture’s growth in total factor productivity after 1990 came from technological progress. Structural transformation also resumed after 1990. 

    Indeed, in 1978, the average labor productivity in China’s nonagricultural sector was six times as high as in the agricultural sector, and therefore one would expect a significant contribution from the labor reallocation. Young (2003) suggests that the reforms in the agricultural sector may have been the most important source of China’s growth during the first two decades of economic reform.

    In Brandt and Zhu (2010), we find that the role of agriculture’s productivity growth diminishes over time. First, as the economy grew, agriculture’s share of valueadded decreased. Second, the marginal contribution of reallocation is a decreasing function of the agricultural productivity level.
    \section*{Growth outside Agriculture: A Tale of Two Sectors}
    Before 1978, the state sector dominated nonagricultural activity. The nonstate sector at that time mainly consisted of collective firms. Urban collectives were confined to producing a small number of consumer goods and providing neighborhood services. Rural collectives were only allowed to manufacture producer goods for the agricultural sector.
    \subsection*{1978 –1988: Rise of the Nonstate Sector}
    In the early 1980s, encouraged by the success of the rural reforms, the Chinese government started two market reforms in the nonagricultural sector. First, a dual-track system was introduced. Second, the central government also devolved economic decision-making powers to lower-level governments and provided them with fiscal incentives. Starting in 1980, a “fiscal contracting system” was implemented that effectively made local governments the “residual claimants” of the enterprises under their control (Qian 1999). 

    Under these reforms, the township and village enterprises based on the old rural collectives flourished, while the state-owned enterprises did not. The success of the agricultural reforms made available to these enterprises a large number of local workers, and the dual-track system allowed them to gain access to capital and raw materials from the markets. The expansion of employment in the nonstate sector was also accompanied by total factor productivity growth.

    Local governments at county level and above sought to improve the economic performance of the state-owned enterprises under their control by implementing a “managerial responsibility system” that linked managers and workers’ income to financial outcomes of the enterprises. The reforms did have some positive effect on productivity, most of which could be attributed to stronger incentives, increased market competition, and better allocation of production inputs.
    
    In the pre-reform period, prices of material inputs were kept artificially low, and so during the reform period, market prices of material inputs rose significantly faster than output prices. While enterprise reforms made industrial state-owned enterprises more efficient, their productivity growth was slower than that of the nonstate enterprises and not fast enough to offset the rising real cost of material inputs. Estimates using gross output rather than value-added in production function may be misleading if the costs of real material inputs are rising. Specifically, let $s_m$ be the share of material inputs in gross output, then 
    \begin{align*}
        \vartriangle ln(TFP_{value-added}) &= [\vartriangle ln(TFP_{gross output})\\
        &-s_m \vartriangle ln(real material input cost)]/(1-s_m).
    \end{align*}
    
    \subsection*{1988 –1998: From Reform without Losers to Inevitable Tradeoffs}
    One reason for the drastic difference in economic performances between the township and village enterprises and the state-owned enterprises is that state-owned enterprises remained under the constraints of government planning for a longer time, unable to sell their products at market prices, although these restrictions diminished over time.

    But the more important difference is the commitment made by the central government to support employment in the state sector. The strategy of letting the nonstate sector grow without downsizing the state sector had the political benefit of minimizing social instability and reducing resistance to reform. Lau, Qian, and Roland (2000) call it “reform without losers”. To avoid laying off workers or shutting down factories, the government usually asked the state-owned banks to bail out loss-making state-owned enterprises. The possibility of bailout created a “soft budget constraint” that further reduced the economic incentives of the state-owned enterprises. While the local governments that ran the township and village enterprises did have political incentives to minimize unemployment and maintain social stability in their communities, these local governments had only weak influence on banks. For example, millions of township and village enterprises went bankrupt when there was a general tightening of credit in 1989. 

    A “reform without losers” strategy still poses tradeoffs. The state-owned enterprises continued to be outperformed by the nonstate sector. Faced with increasing competition from the more efficient nonstate firms and without significant productivity growth, the financial condition of the state-owned firms deteriorated. Resources needed to support the state-owned enterprises increased steadily between 1986 and 1993. Nonperforming loans in the state banking system increased rapidly, and the creation of money to make these loans was leading to chronic high inflation.

    By 1994, It had become clear that the strategy of “reform without losers” could no longer be sustained. In 1995, the Chinese government reduced its commitment. Many small-scale state-owned enterprises were allowed to go bankrupt or be privatized through management buyouts. More diversified ownership forms were also introduced within the state sector. Some of the large-scale state-owned enterprises were converted into shareholding companies, with a majority of shares controlled by the state. 

    \subsection*{1998–2007: Privatization and Trade Liberalization}
    The 15th Congress of the Chinese Communist Party held in 1997 formally sanctioned ownership reforms of the state-owned firms and also legalized the development of private enterprises.  Private enterprises grew rapidly. Collective enterprises such as township and village enterprises lost their edge, some were closed and many of them were privatized, also in the form of management buyouts. China’s government also started to cut tariffs, broadened trade rights, and liberalized its regime for foreign direct investment as part of the lead-up to joinning WTO.

    The combination of privatization and trade liberalization had strong effects on productivity growth in both the state and nonstate sectors. In the manufacturing sector, productivity growth during this period is even higher. However, China’s nontradable sectors—primarily construction and services—have faced much less international competition. There have also been significant barriers to entry of private and foreign-invested firms into service industries, and significant barriers to exit of state-owned enterprises in services. And in 2007, the state sector still accounted for 77 percent of total urban employment in services, in contrast to 15 percent in manufacturing. 

    \section*{The Future of China’s Economic Growth}
    Experiences from other economies, especially the East Asian economies such as Japan, Korea, and Taiwan, suggest that periods of extremely rapid growth eventually slow down.

    we can decompose China’s GDP per capita relative to that of the United States into four ingredients. China’s labor force participation and capital/output ratios are above U.S. levels, while China’s relative level of human capital has risen somewhat over time. But clearly, the growth of China’s relative GDP per capita is mainly driven by the growth of China’s relative total factor productivity. Many other economies in Eastern Europe and Latin America also had economic reforms, but their growth performances are nowhere near the performance achieved by China. One potential explanation is simply China’s backwardness at the start of economic reform in 1978, which increased China’s potential for catch-up growth. When China started economic reform in 1978, its aggregate total factor productivity was less than 3 percent of the U.S. level, much lower than Mexico and the economies in Eastern Europe and South America. 

    Compare China’s growth experience with three other East Asian economies: Japan, Korea, and Taiwan. In 1950, Japan’s total factor productivity was 56 percent of the U.S. level; by 1975, Japan’s was at 83 percent of the U.S. level. But since then, Japan’s relative total factor productivity has somewhat fallen back. In 1965, Korea’s total factor productivity was 43 percent of the U.S. level; by 1990, it had reached 63 percent of the U.S. level. After 1990, Korea’s relative productivity has continued to converge with the U.S. level, but at a much slower rate of about 0.24 percent per year. In 1965, Taiwan’s total factor productivity was 50 percent of the U.S. level; by 1990, it had reached 80 percent of that in the United States. Since then, Taiwan’s relative total factor productivity has continued to converge, but (like Korea) at a much slower rate. Even if China can replicate this extraordinary growth performance for another two decades, its productivity level would still be only 40 percent of the frontier U.S. level—still below the level of Japan in the 1950s or South Korea and Taiwan in the 1960s. 

    China’s economy still has large opportunities for raising productivity growth through reducing the still-existing distortions and inefficiencies in its production. Despite many years of financial sector reforms, China’s banking sector is still dominated by the state-controlled banks that lend disproportionately to local government projects and to firms in the state sector. Protected by barriers to entry of private and foreign firms, statecontrolled firms continue to enjoy substantial monoploy rights and profits in industries ranging from energy, transportation, and telecomunication to banking, entertainment, education, and health care. 

    In the last three and half decades, China’s leaders have chosen to carry out economic reform without political reform or the establishment of rule of law. They have implemented reforms in a piecemeal fashion that usually provided benefits to key interest groups within the state sector. While this approach has helped to reduce political resistence to economic reform, it has also resulted in corruption and income inequality in addition to economic distortions. 




\end{document}